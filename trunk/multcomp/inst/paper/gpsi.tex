
\documentclass[12pt]{article}
\usepackage{a4wide}
../paper/header.tex

\title{Generalized Simultaneous Inference \\ in Parametric Models}
\author{\textbf{Torsten Hothorn} \\
Institut f{\"u}r Statistik \\
Ludwig-Maximilians-Universit{\"a}t M{\"u}nchen \\
Ludwigstra{\ss}e 33, D--80539 M{\"u}nchen, Germany\\
\and \textbf{Frank Bretz} \\
Biostatistics \& Statistical Reporting, WSJ-27.1.005\\
Novartis Pharma AG \\
CH-4002 Basel, Switzerland \\
\and \textbf{Peter Westfall} \\
Texas Tech University \\
Lubbock, TX 79409, U.S.A}

\begin{document}

\maketitle

\begin{abstract}
Abstract
\end{abstract}

\textbf{Key words}: multiple tests, multiple comparisons, confidence intervals,
adjusted $p$-values, multivariate normal distribution, robust statistics.

\newpage

\section{Introduction}

\section{Model and Estimation} \label{model}

Let $\M(\Z_n, \vartheta, \eta)$ denote a (semi-)parametric statistical model.
The vector of $n$ observations is denoted by $\Z_n = (\Z_1, \dots, \Z_n)$. 
The model is equiped with fixed
but unknown parameters of interest $\vartheta \in \R^p$ 
and other (random or nuisance) parameters $\eta$. We are provided 
with an estimate $\hat{\vartheta} \in \R^p$ of the parameter 
vector $\vartheta$ and are interested in a linear
function $f_{\K, \m}(\vartheta) = \K \vartheta - \m$
of the parameters defined by finite constants $\K \in \R^{k, p}$
and $\m \in \R^k$.

\paragraph{Model assumptions.}

We assume that the parameter estimates $\hat{\vartheta}$ converge in law
to a $p$-dimensional multivariate normal distribution:
\begin{eqnarray} \label{normality}
\hat{\vartheta} \cL \N_p\left(\vartheta, \Sigma\right)
\end{eqnarray}
with constant and finite covariance matrix $\Sigma \in \R^{p,p}$.
Since the covariance matrix is usually unknown, we assume 
that we are given a consistent estimate $\hat{\Sigma} \cP \Sigma$.

\paragraph{Limiting distribution of $f_{\K,\m}(\hat{\vartheta})$.}

The linear function $f_{\K, \m}$ is continuous and differentiable in
$\vartheta$ with Jacobian $\K$ and thus, by Cram{\'e}r's Theorem, 
the linear function $f$ of the parameter estimates
converges in law to a (possibly singular) $k$-dimensional normal distribution:
\begin{eqnarray} \label{dist_f}
f_{\K, \m}(\hat{\vartheta}) = \K \hat{\vartheta} - \m 
\cL \N_k\left(\K \vartheta - \m, \K \Sigma \K^\top\right).
\end{eqnarray}
Furthermore, it follows from the same theorem that we can 
consistently estimate the unknown covariance matrix $\K \Sigma \K^\top$ utilizing
the consistent estimate $\hat{\Sigma}$ of the covariance matrix $\Sigma$:
\begin{eqnarray*}
\hat{\S} := \K \hat{\Sigma} \K^\top \cP \K \Sigma \K^\top =: \S.
\end{eqnarray*}

\paragraph{Test statistic and limiting distribution.}

Let $\T := \hat{\D} (\K \hat{\vartheta} - \m) \in \R^k$ denote
the $k$-dimensional standardized test statistic where $\hat{\D}$ is the diagonal
matrix of standard deviations
\begin{eqnarray*}
\hat{\D} := \text{diag}\left(\text{diag}(\hat{\S})^{-1/2}\right) \cP 
\text{diag}\left(\text{diag}(\S)^{-1/2}\right) =: \D \in \R^{k,k}
\end{eqnarray*}
In order to show that also $\T$ converges in law to a $k$-dimensional multivariate
normal distribution
\begin{eqnarray*}
\T \cL \N_k\left(\D(\K \vartheta - \m), \Cor := \D \S \D^\top\right)
\end{eqnarray*}
we utilize the Cram{\'e}r-Wold device and show that for some $\uu \in \R^k$ 
the linear function $\uu^\top \T$ is univariate normal. Consider
\begin{eqnarray*}
\uu^\top \T & = & \uu^\top \hat{\D} (\K \hat{\vartheta} - \m) \\
& = & \left(u_1 \hat{\S}_{11}^{-1/2}, \dots, u_k \hat{\S}_{kk}^{-1/2}\right)^\top(\K \hat{\vartheta} - \m).
\end{eqnarray*}
Since $\hat{\S}$ is a consistent estimate of $\S$ it 
is follows that $u_j \hat{\S}_{jj}^{-1/2} \cP u_j \S_{jj}^{-1/2}$.
Together with (\ref{dist_f}) we can apply Slutsky's theorem to show that
\begin{eqnarray*}
u_j \hat{\S}_{jj}^{-1/2} (\K \hat{\vartheta} - \m)_j & \cL &  
\N_1\left(u_j \S_{jj}^{-1/2} (\K \vartheta - \m)_j, u_j^2 \S_{jj}^{-1} \S_{jj}\right) \\
& = & \N_1\left(u_j \S_{jj}^{-1/2} (\K \vartheta - \m)_j, u_j^2\right)
\end{eqnarray*}
Because the above holds for every $\uu \in \R^k$, the asymptotic multivariate
normaliy of $\T$ follows from the Cram{\'e}r-Wold device. Note that
all elements of $\T$ have mean zero and variance $1$. In summary,
the vector of test statistics $\T$ converges in law to a multivariate
normal distribution with correlation matrix 
$\Cor := \D \S \D^\top \in \R^{k,k}$.

\section{Simultaneous Inference}

Under the global hypothesis
\begin{eqnarray*}
H_0: \K \vartheta = \m
\end{eqnarray*}
it has been shown in Section~\ref{model} that 
$\T \cL \N_k(0, \Cor)$ holds true. At this point it is worth
considering two special cases. A stronger assumption than asymptotic normality
of $\hat{\vartheta}$ (\ref{normality}) is exact normality, i.e., 
$\hat{\vartheta} \sim \N_p(\vartheta, \Sigma)$. Under the unrealistic
further assumption of the covariance matrix $\Sigma$ being known, it
follows by standard arguments that $\T \sim \N_k(0, \Cor)$. 
When $\Sigma$ is unknown (the typical situation of linear models
with normal iid errors and constant variance), the
exact distribution of $\T$ is a $k$-dimensional multivariate $\mathcal{T}$
distribution with $df$ degrees of freedom (df $ = n - p - 1$ for linear models).
\marginpar{Ref}

Now, the distribution function of $\T$ can be calculated or approximated
by evaluating the $k$-dimensional exact or limiting distributions:
\begin{eqnarray*}
& & \Prob(\max(\T) \le t)  \cong\int\limits_{-\infty}^t \cdots \int\limits_{-\infty}^t 
\varphi_k(x_1, \dots, x_k; \Cor, \text{df}) \, dx_1 \cdots dx_k \quad \text{("less")}\\
& & \Prob(\max(|\T|) \le t)  \cong  \int\limits_{-t}^t \cdots \int\limits_{t}^t 
\varphi_k(x_1, \dots, x_k; \Cor, \text{df}) \, dx_1 \cdots dx_k \quad \text{("two sided")}\\
& & \Prob(\min(\T) \ge t) \cong  \int\limits_{t}^\infty \cdots \int\limits_{t}^\infty 
\varphi_k(x_1, \dots, x_k; \Cor, \text{df}) \, dx_1 \cdots dx_k \quad \text{("greater")}\\
\end{eqnarray*}
where $\varphi_k$ is either the density function of the limiting multivariate
normal (df $ = 1$ and `$\approx$') or exact multivariate $\mathcal{T}$-distribution (df $ > 1$
and `"="').
Since $\Cor$ is usually unknown, we plug-in a consistent estimate
$\hat{\Cor} := \hat{\D} \hat{\S} \hat{\D}^\top$.

\paragraph{Simultaneous tests.}

An adjusted $p$-value for the $j$th partial hypothesis 
\begin{eqnarray*}
H_0^j: (\K \vartheta)_j = \m_j
\end{eqnarray*}
is given by $1 - \Prob(\max(|\T|) \le |t_j|)$ when the test statistic 
$\tt  = (t_1, \dots, t_k)$
has been observed. The global $p$-value for $H_0$ is $1 - \Prob(\max(|\T|) \le \max|t_j|)$.
Alternatively, a quadratic form based on the Moore-Penrose inverse $\hat{\S}^+$ of 
$\hat{\S}$ or the usual $F$-test (assuming exact normality of $\hat{\vartheta}$) 
can be utilized to construct a global test:
\begin{eqnarray*}
X^2 & = & (\K \hat{\vartheta} - \m)^\top \hat{\S}^+ (\K \hat{\vartheta} - \m) \cL \chi^2_{\Rg(\hat{\S})} \\
F & = &  \frac{X^2}{\Rg(\hat{\S})} \sim \F_{\text{df}, \Rg(\hat{\S})}
\end{eqnarray*}

\paragraph{Simultaneous confidence sets.}

A simultaneous $(1 - 2\alpha) \times 100\%$ 
confidence set for $\K \vartheta$ is given by 
$\K \hat{\vartheta} \pm q_\alpha \text{diag}(\hat{\S})^{-1/2}$
where $q_\alpha$ is the $1 - \alpha$ 
quantile of the distribution of $\T$ 
such that $\Prob(\max{\T} \le q_\alpha) \ge 1 - \alpha$.

\section{Applications}

The framework described above is rather general and applicable 
to a wide range of models since most estimates are at least asymptotically
normal and a consistent estimate of their covariance matrix
is usually available and readily implemented. Most prominent
representatives are models fitted by maximum likelihood, M-estimation,
restricted maximum likelihood etc. \marginpar{more!}
It should be noted that there we didn't make assumptions on the form
of $\K$ whatsoever and thus a rather straightforward construction 
principle for new simultaneous inference procedures is available. In
what follows, we assume $\m = 0$ but this is not a restriction.
In the following we will highlight a subjectiv selection of 
some special cases of practical importance.

\paragraph{One-way ANOVA.}

Consider a one-way ANOVA model for a factor measured at three levels
with a continuous response and normal errors $\varepsilon_{ij} \sim \N_1(0, \sigma^2)$, 
i.e., the model
\begin{eqnarray*}
Y_{ij} = \mu + \alpha_{j} + \varepsilon_{ij}, j = 1, 2, 3, i = 1, \dots, n_j.
\end{eqnarray*}
The model is overparameterized and thus only the parameters
$\vartheta = (\mu, \alpha_2 - \alpha_1, \alpha_3 - \alpha_1)$ are 
estimated by classical least-squares. (Note that other contrasts can be used
but theses so-called `treatment contrasts' are the most prominent ones and,
for example, the default in \RR.) 
Most of the classical approaches
to multiple comparisons can be embedded into this simple framework and 
the two most prominent ones are Dunnett's many-to-one procedure and
Tukey's honest significant differences. For Dunnett's procedures,
the differences of all coefficients with $\alpha_1$ under test, i.e.,
a linear function
\begin{eqnarray*}
\K_\text{Dunnett} = (0, \diag(2))
\end{eqnarray*}
with so called `contrasts' $\K_\text{Dunnett} \vartheta = (\alpha_2 - \alpha_1, \alpha_3 - \alpha_1)$.
For Tukey's honest significant differences the interest is in inference on
all possible differences of the model parameters $\alpha_1, \alpha_2, \alpha_3$, i.e., 
the matrix
\begin{eqnarray*}
\K_\text{Tukey} = \left( 
\begin{array}{rrr} 0 & 1 & 0 \\
0 & 0 & 1 \\
0 & 1 & -1 
\end{array} \right)
\end{eqnarray*}
with $\K_\text{Tukey} \vartheta = (\alpha_2 - \alpha_1, \alpha_3 - \alpha_1, \alpha_2 - \alpha_3)$.
Many other contrasts have been studies in the past, such as Williams, McDermott, Marcus,
Sequential, Changepoints etc. pp. \marginpar{Ref!} and all can be formulated in terms
of a linear function $\K \vartheta$.

Under the standard assumptions of ANOVA models, i.e., iid normal errors with constant
variance, the vector of test statistics $\T$ follows an exact $\mathcal{T}$-distribution and
thus, simultaneous tests and confidence intervals are `exact' as well.

\paragraph{GLM, (N)LME and survival models.}

In generalized linear models, the exact distribution of the parameter estimates
is usually unknown and thus the asymptotic normal distribution is the basis for all
inference procedures. When we are interested in inference about model
parameters corresponding to levels of a certain factor, the same multiple 
comparison procedures as sketched above are available. 

Linear and non-linear mixed effects models fitted by REML provide
the data analyst with asymptotically normal errors and a consistent
covariance matrix as well so that all assumptions of our framework
are met and one can set up simultaneous inference procedures
for these models as well.

The same is true for the Cox model or other parametric survival models
such as the Weibull model.

\paragraph{Robust simultaneous inference.}

An interesting possibility is to move to robust variants of the above
mentioned models. One obvious possibility is the application
of sandwich estimators $\hat{\Sigma}$ of the covariance matrix $\Sigma$, 
for example when the variance homogenity in linear models is violated.
An even more robust approach is to use robust linear models, for example
fitted using MM.

The reader is referred to Section~\ref{illustrations} for some practical
instances of these ideas.

\section{Implementation}

\input{illustrations}

\input{trees}

\bibliographystyle{plainnat}
\bibliography{references}

\end{document}

