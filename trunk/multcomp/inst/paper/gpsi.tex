%% ToDo
%% Introduction: more elaborate formulations
%% add some references to recently published technology
%% Section 2  / 3: Model & Inference
%% check
%% reference for multivariate T distribution and
%% its applications
%% Section 4: Applications
%% references for M, S, REML etc. estimation
%% Section 5: Implementation
%% check
%% Section 6: Illustrations:
%% check
%% Section 7: Discussion
%% some comments on why we saved the world today.
%% maybe some more references


\documentclass[12pt]{article}
%%%%%%%%%%%%%%%%%%%%%%%%%%%%%%%%%%%%%%%%%%%%%%%%%%%%%%%%%%%%%%%%%%%%%%%%%%%%%%%%%%%%%%%%%%%%%%%%%%%%%%%%%%%%%%%%%%%%%%%%%%%%%%%%%%%%%%%%%%%%%%%%%%%%%%%%%%%%%%%%%%%%%%%%%%%%%%%%%%%%%%%%%%%%%%%%%%%%%%%%%%%%%%%%%%%%%%%%%%%%%%%%%%%%%%%%%%%%%%%%%%%%%%%%%%%%
\usepackage{a4wide}
\usepackage[lists,heads]{endfloat}

%TCIDATA{OutputFilter=Latex.dll}
%TCIDATA{Version=5.00.0.2606}
%TCIDATA{<META NAME="SaveForMode" CONTENT="1">}
%TCIDATA{BibliographyScheme=BibTeX}
%TCIDATA{LastRevised=Friday, January 25, 2008 13:26:30}
%TCIDATA{<META NAME="GraphicsSave" CONTENT="32">}

../paper/header.tex
\hypersetup{  pdftitle = {Simultaneous Inference in General Parametric Models},
  pdfsubject = {Manuscript},
  pdfauthor = {Torsten Hothorn and Frank Bretz and Peter Westfall},
  colorlinks = {true},
  linkcolor = {blue},
  citecolor = {blue},
  urlcolor = {red},
  hyperindex = {true},
  linktocpage = {true},
}

%%\input{tcilatex}

\begin{document}

\title{Simultaneous Inference \\
in General Parametric Models}
\author{\textbf{Torsten Hothorn} \\
%EndAName
Institut f{\"u}r Statistik \\
Ludwig-Maximilians-Universit{\"a}t M{\"u}nchen \\
Ludwigstra{\ss }e 33, D--80539 M{\"u}nchen, Germany\\
\and \textbf{Frank Bretz} \\
%EndAName
Statistical Methodology, Clinical Information Sciences\\
Novartis Pharma AG \\
CH-4002 Basel, Switzerland \\
\and \textbf{Peter Westfall} \\
%EndAName
Texas Tech University \\
Lubbock, TX 79409, U.S.A}
\maketitle

\begin{abstract}
Simultaneous inference is a common problem in many areas of application. If
multiple null hypotheses are tested simultaneously, the probability of
rejecting erroneously at least one of them increases beyond the
pre-specified significance level. Simultaneous inference procedures have to
be used which adjust for multiplicity and thus control the overall type I
error rate. In this paper we describe simultaneous inference procedures in
general parametric models, where the experimental questions are specified
through a linear combination of the model parameters. The framework
described here is quite general and extends the canonical theory of multiple
comparison procedures in ANOVA models to linear regression problems,
generalized linear models, linear mixed effects models, the Cox model,
robust linear models, etc. Several examples using a variety of different
statistical models illustrate the breadth of the results. For the analyses
we use the \RR{} add-on package \Rpackage{multcomp}, which provides a
convenient interface to the general approach adopted here.
\end{abstract}

\thispagestyle{empty} \setcounter{page}{0}

\textbf{Key words}: multiple tests, multiple comparisons, simultaneous
confidence intervals, \newline
adjusted $p$-values, multivariate normal distribution, robust statistics.

\newpage

\section{Introduction}

Multiplicity is an intrinsic problem of any simultaneous inference. If each
of $k$, say, null hypotheses is tested at nominal level $\alpha$, the
overall type I error rate can be substantially larger than $\alpha$. That
is, the probability of at least one erroneous rejection is larger than $%
\alpha$ for $k \geq 2$. Common multiple comparison procedures adjust for
multiplicity and thus ensure that the overall type I error remains below the
pre-specified significance level $\alpha$. Examples of such multiple
comparison procedures include Dunnett's many-to-one comparisons, Tukey's
all-pairwise differences, sequential pairwise contrasts, comparisons with
the average, changepoint analyses, dose-response contrasts, etc. These
procedures are all well established for classical regression and ANOVA
models allowing for covariates and/or factorial treatment structures with
i.i.d.~normal errors and constant variance, see \cite{Bretzetal2008} and the
references therein. For a general reading on multiple comparison procedures
we refer to \cite{HochbergTamhane1987} and \cite{Hsu1996}.

In this paper we aim at a unified description of simultaneous inference
procedures in parametric models with generally correlated parameter
estimates. Each individual null hypothesis is specified through a linear
combination of the model parameters and we allow for $k$ of such null
hypotheses to be tested simultaneously. The general framework described here
extends the current canonical theory with respect to the following aspects:
(i) model assumptions, such as normality and homoscedasticity are relaxed,
thus allowing for simultaneous inference in generalized linear models,
mixed-effects models, survival models, etc.; (ii) arbitrary linear functions
of the parameters are allowed, not just contrasts of means in AN(C)OVA
models; (iii) computing the reference distribution is feasible for arbitrary
designs, especially unbalanced designs; and (iv) a unified implementation is
provided which allows for a fast transition of the theoretical results to
the desks of data analysts interested in simultaneous inferences for
multiple hypotheses.

Accordingly, the paper is organized as follows. Section~\ref{model} defines
the general model and obtains the asymptotic or exact distribution of linear
functions of model parameters under rather weak conditions. In Section~\ref%
{siminf} we describe the framework for simultaneous inference procedures in
general parametric models. An overview about important applications of the
methodology is given in Section~\ref{applications} followed by a short
discussion of the software implementation in Section~\ref{implementation}.
Most interesting from a practical point of view is Section~\ref%
{illustrations} where we analyze four rather challenging problems with the
tools developed in this paper.

\section{Model and Estimation}

\label{model}

In this section we introduce the underlying model assumptions and derive
some asymptotic results necessary in the subsequent sections. The results
from this section form the basis for the simultaneous inference procedures
described in Section~\ref{siminf}.

Let $\M(\Obs, \vartheta, \eta)$ denote a (semi-)parametric statistical
model. The set of $n$ observations is described by $\Obs = (\Obs_1, \dots,
\Obs%
_n)$. The model contains fixed but unknown parameters of interest $\vartheta
\in \R^p$ and other (random or nuisance) parameters $\eta$. We are
interested in the linear functions $\K \vartheta$ of the parameter vector $%
\vartheta$ as specified through the constant matrix $\K \in \R^{k, p}$.

Assume that we are given an estimate $\hat{\vartheta}_n \in \R^p$ of the
parameter vector $\vartheta$. In what follows we describe the underlying
model assumptions, the limiting distribution of $\K\hat{\vartheta}_n$ as well
as the corresponding test statistics and their limiting joint distribution.

\paragraph{Large Sample Assumptions.}

Suppose $\K^{\prime }=[a_{1}|\ldots |a_{k}]$ with $a_{i}\in \R^{p}$. \ The
parametric functions of interest to the researcher are $\psi
_{j}=a_{j}^{\prime }\vartheta $. Define $\Psi =\K\vartheta $,  $\hat{\psi}%
_{j}=a_{j}^{\prime }\hat{\vartheta}$,  $\widehat{\Psi
}:=\K\hat{\vartheta}.$%
\ Supposing an estimated covariance matrix  $\hat{\Sigma}$ of
$\hat{\vartheta%
}$ is available, define the normalized statistics $z_{j,n}=(\hat{\psi}%
_{j}-\psi _{j})/(a_{j}^{\prime }\hat{\Sigma}a_{j})^{1/2}$. \ Let $%
Z_{n}^{\prime }=(z_{1,n},\ldots ,z_{k,n}),$ $D_{n}=diag(\K\hat{\Sigma}\K%
^{\prime }),$ and $\boldsymbol{\hat{\rho}}_{n}=D_{n}^{-1/2}\K\hat{\Sigma}\K%
^{\prime }D_{n}^{-1/2}$. \ We assume sufficient regularity conditions on the
model such that, as $n\rightarrow \infty $,%
\[
\boldsymbol{\hat{\rho}}_{n}\rightarrow _{p}\boldsymbol{\rho },
\]%
where $\boldsymbol{\rho }$ is non-negative definite; and that
\[
Z_{n}\cL\N_{k}\left( 0,\boldsymbol{\rho }\right) .
\]%
In this case we can treat the vector $Z_{n}$ as having an approximately
normal distribution with mean vector zero and covariance matrix  $%
\boldsymbol{\hat{\rho}}_{n}$,


\bigskip 

(Now some comment on the notion that these assumptions are justified in
typical GLMs, including logistic rgeression etc., again under suitable
regularity conditions. \ Then much of the following can go ) \bigskip 

\bigskip 

\paragraph{TH new:}

We assume that we are provided with an estimate $\hat{\vartheta}_n$ 
of the model parameters
$\vartheta$ and a multivariate central limit theorem stating
\begin{eqnarray} \label{assume}
\sqrt{n} (\hat{\vartheta}_n - \vartheta) \cL \N_p(0, \S(\vartheta))
\end{eqnarray}
where $\S(\vartheta) \in \R^{p,p}$ is a, typically unknown, covariance
matrix.
In the notation of \cite{Serfling1980}, this is equivalent to
\begin{eqnarray*}
\hat{\vartheta}_n - \vartheta \text{ is } \AN_p(0, n^{-1} \S(\vartheta))
\end{eqnarray*}
\citep[Section 1.5.5][]{Serfling1980}. By Theorem 3.3.A in
\cite{Serfling1980} the linear function 
\begin{eqnarray*}
\K (\hat{\vartheta}_n - \vartheta) \text{ is } \AN_k(0, n^{-1} \K \S(\vartheta) \K^\top)
\end{eqnarray*}
and, by the same theorem, with $\D(\vartheta) = \text{diag}(\K \S(\vartheta) \K^\top)^{-1/2}$
\begin{eqnarray} \label{as}
\D(\vartheta) \K (\hat{\vartheta}_n - \vartheta) \text{ is } 
\AN_k(0, n^{-1} \underbrace{\D(\vartheta) \K \S(\vartheta) \K^\top
\D(\vartheta)^\top}_{=: \boldsymbol{\rho}(\vartheta)})
\end{eqnarray}


We assume that an estimate $\hat{\Sigma}(\hat{\vartheta}_n) \cP 0$ 
of the covariance matrix $\hat{\Sigma}(\hat{\vartheta}_n) =
\text{cov}(\hat{\vartheta}_n)$ of $\hat{\vartheta}_n$ is available such that
the empirical version
\begin{eqnarray*}
\hat{\D}(\hat{\vartheta}_n) = \text{diag}(\K \hat{\Sigma}(\hat{\vartheta}_n) \K^\top)^{-1/2}
\end{eqnarray*}
of $\D$ satisfies
\begin{eqnarray*}
\hat{\boldsymbol{\rho}}(\hat{\vartheta}_n) = \hat{\D}(\hat{\vartheta}_n) \K
\hat{\Sigma}(\hat{\vartheta}_n) \K^\top \hat{\D}(\hat{\vartheta}_n)^\top \cP 
\boldsymbol{\rho}(\vartheta)
\end{eqnarray*}
and $\hat{\D}(\hat{\vartheta}_n) \cP \sqrt{n} \D(\vartheta)$.
Then, by Slutzky's Theorem \citep[Theorem 1.5.4][]{Serfling1980}, 
\begin{eqnarray*}
\T_n := \hat{\D}(\hat{\vartheta}_n) \K (\hat{\vartheta}_n - \vartheta) \cL
\D(\vartheta) \K (\hat{\vartheta}_n - \vartheta) \text{ which is } \AN_k(0,
n^{-1} \boldsymbol{\rho}(\vartheta))
\end{eqnarray*}
With (\ref{as}), and in the convential notation, we have
\begin{eqnarray*}
\T_n \cL \N_k(0, \boldsymbol{\rho}(\vartheta))
\end{eqnarray*}
In this case we can treat the vector $\T_{n}$ as having an approximately
normal distribution with mean vector zero and correlation matrix  $%
\boldsymbol{\hat{\rho}}(\hat{\vartheta}_n)$.

The imposed assumption (\ref{assume}) is rather standard, and, for example,
holds for all ML estimates with $\S(\vartheta)$ being the inverse of the
Fisher information matrix.



Since in the following we only assume that the parameters estimates are
asymptotically normally distributed with a consistent estimate of the
associated covariance matrix being available, our framework covers a large
class of statistical models, including linear regression and ANOVA models,
generalized linear models, linear mixed effects models, the Cox model,
robust linear models, etc. Standard software packages can be used to fit
such models and obtain the estimates $\hat{\vartheta}$ and $\hat{\Sigma}$
which are essentially the only two quantities that are needed for what
follows. It should be noted that the model parameters $\vartheta $ are not
necessarily means or differences of means in AN(C)OVA models. Also, we do
not restrict our attention to contrasts of such means, but allow for any set
of constants leading to the linear functions $\K\vartheta $ of interest.
Specific examples for $\K$ and $\vartheta $ will be given later in Section~%
\ref{illustrations}.

\paragraph{Limiting distribution of $\K\hat{\protect\vartheta}$.}

The linear function $\K \vartheta$ is continuous and differentiable in $%
\vartheta$ and thus, by Theorem 5.1.5 in \cite{Lehmann1999} and standard
arguments for linear transformations of multivariate normal variables, the
linear function $\K \vartheta$ of the parameter estimates converges in law
to a (possibly singular) $k$-dimensional normal distribution: 
\begin{eqnarray}  \label{dist_f}
\sqrt{n} \left(\K \hat{\vartheta} - \K \vartheta\right) \cL \N_k\left(0, \K %
\Sigma \K^\top\right).
\end{eqnarray}
Furthermore, because the quadratic form $\K \hat{\Sigma} \K^\top$ is
continuous and, by assumption, $\hat{\Sigma} \cP \Sigma$, it follows from
Theorem 5.1.1 in \cite{Lehmann1999} that we can consistently estimate the
unknown covariance matrix $\K \Sigma \K^\top$ utilizing a consistent
estimate $\hat{\Sigma}$ of the covariance matrix $\Sigma$: 
\begin{eqnarray*}
\hat{\S } := \K \hat{\Sigma} \K^\top \cP \K \Sigma \K^\top =: \S .
\end{eqnarray*}
Anticipating that asymptotically non-degenerate linear functions of the data
are of interest, we assume that the diagonal elements of $S$ are strictly
positive.

\paragraph{A multivariate statistic and its limiting distribution.}

Let $\T := \sqrt{n} \hat{\D} (\K \hat{\vartheta} - \K \vartheta) \in \R^k$
denote the $k$-dimensional standardized statistic, where $\hat{\D}$ is the
diagonal matrix of standard deviations, that is, 
\begin{eqnarray*}
\hat{\D} := \text{diag}\left(\text{diag}(\hat{\S })^{-1/2}\right) \cP \text{%
diag}\left(\text{diag}(\S )^{-1/2}\right) =: \D \in \R^{k,k}.
\end{eqnarray*}
Note that $\T$ converges in law to a $k$-dimensional multivariate normal
distribution, that is, 
\begin{eqnarray*}
\T \cL \N_k\left(0, \Cor\right),
\end{eqnarray*}
where $\Cor := \D \S \D^\top$. To see this we utilize the Cram{\'e}r-Wold
device \citep[e.g., Theorem 5.1.8
in][]{Lehmann1999} and show that for any $\uu \in \R^k$ the linear function $%
\uu^\top \T$ is univariate normal. Consider 
\begin{eqnarray*}
\uu^\top \T & = & \uu^\top \sqrt{n} \hat{\D} (\K \hat{\vartheta} - \K %
\vartheta) \\
& = & \sqrt{n} \left(u_1 \hat{\S }_{11}^{-1/2}, \dots, u_k \hat{\S }%
_{kk}^{-1/2}\right)^\top(\K \hat{\vartheta} - \K \vartheta).
\end{eqnarray*}
Since $\hat{\S }$ is a consistent estimate of $\S $ it follows that $u_j 
\hat{\S }_{jj}^{-1/2} \cP u_j \S _{jj}^{-1/2}$. Together with (\ref{dist_f})
we can apply Slutsky's theorem \citep[Theorem 2.3.3
in][]{Lehmann1999} to show that 
\begin{eqnarray*}
\sqrt{n} u_j \hat{\S }_{jj}^{-1/2} (\K \hat{\vartheta} - \K \vartheta)_j & %
\cL & \N_1\left(0, u_j^2 \S _{jj}^{-1} \S _{jj}\right) \\
& = & \N_1\left(0, u_j^2\right).
\end{eqnarray*}
Because the above holds for every $\uu \in \R^k$, the asymptotic
multivariate normality of $\T$ follows from the Cram{\'e}r-Wold device.
Based on the statistic $\T$ we will derive test statistics and utilize its
limiting distribution as reference distribution when constructing
simultaneous inference procedures in the next section.

\section{Global and Simultaneous Inference}

\label{siminf}

Based on the results from Section~\ref{model}, we now focus on the
derivation of suitable inference procedures. We start considering the
general linear hypothesis \citep{Searle1971} 
\begin{eqnarray*}
H_0: \K \vartheta = \m.
\end{eqnarray*}
Under the conditions of $H_0$ it follows from Section~\ref{model} that $\T = 
\sqrt{n} \hat{\D} (\K \hat{\vartheta} - \m) \cL \N_k(0, \Cor)$. This
limiting distribution will now be used as the reference distribution when
constructing the inference procedures. The global hypothesis $H_0$ can be
tested using standard global tests, such as the $F$- or the $\chi^2$-test.
An alternative approach is to use maximum tests, as explained in Subsection~%
\ref{global}. Note that a small global $p$-value leading to a rejection of $%
H_0$ does not give further indication about the nature of the significant
result. Therefore, one is often interested in the individual null hypotheses 
\begin{eqnarray*}
H_0^j: (\K\vartheta)_j = \m_j.
\end{eqnarray*}
(Note that $H_0 = \bigcap_{j = 1}^k H_0^j$.) Testing the hypotheses set $%
\{H_0^1, \ldots, H_0^k\}$ simultaneously thus requires the individual
assessments while maintaining the familywise error rate, as discussed in
Subsection~\ref{simtest}

At this point it is worth considering two special cases. A stronger
assumption than asymptotic normality of $\hat{\vartheta}$ (\ref{normality})
is exact normality, i.e., $\sqrt{n} (\hat{\vartheta} - \vartheta) \sim \N%
_p(0, \Sigma)$. If the covariance matrix $\Sigma$ is known, it follows by
standard arguments that $\T \sim \N_k(0, \Cor)$. Otherwise, if $\Sigma =
\sigma^2 \A$, where $\A$ is fixed and known but $\sigma^2$ is an unknown
constant (which is the typical situation of linear models with normal
i.i.d.~errors and constant variance), the exact distribution of $\T$ is a $k$%
-dimensional multivariate $t_p(\nu, \Cor)$ distribution with $\nu$ degrees
of freedom ($\nu = n - p - 1$ for linear models) see \citep{Tong1990}.

\subsection{Global Inference}

\label{global} %\paragraph{Global tests.}

The $F$- and the $\chi^2$-test are classical approaches to assess the global
null hypothesis $H_0$. Standard results ensure that 
\begin{eqnarray*}
X^2 & = & (\K \hat{\vartheta} - \m)^\top \hat{\S }^+ (\K \hat{\vartheta} - \m%
) \cL \chi^2(\Rg(\hat{\S })) \quad \text{for } \hat{\vartheta} \cL \N%
_p(\vartheta, \Sigma), \text{ and } \\
F & = & \frac{X^2}{\Rg(\hat{\S })} \sim \F(\nu, \Rg(\hat{\S })) \quad \text{%
for } \hat{\vartheta} \sim \N_p(\vartheta, \Sigma),
\end{eqnarray*}
where $\Rg(\hat{\S })$ and $\nu$ are the corresponding degrees of freedom
and $\hat{\S }^+$ is the Moore-Penrose inverse of $\hat{\S }$.

Another suitable scalar test statistic for testing the global hypothesis $H_0
$ is to consider the maximum of the individual test statistics $T_1, \dots,
T_k$ of the statistic $\T = (T_1, \dots, T_k)$, leading to a max-$t$ type
test statistic $\max(|\T|)$. The distribution of this statistic under the
conditions of $H_0$ can be handled through the $k$-dimensional distribution 
\begin{eqnarray}  \label{maxt}
\Prob(\max(|\T|) \le t) \cong \int\limits_{-t}^t \cdots \int\limits_{t}^t
\varphi_k(x_1, \dots, x_k; \Cor, \nu) \, dx_1 \cdots dx_k
\end{eqnarray}
for some $t \in \R$, where $\varphi_k$ is the density function of either the
limiting $k$-dimensional multivariate normal (with $\nu = \infty$ and the `$%
\approx$' operator) or the exact multivariate $t_p(\nu, \Cor)$-distribution
(with $\nu < \infty$ and the `$=$' operator). Since $\Cor$ is usually
unknown, we plug-in the consistent estimate $\hat{\Cor} := \hat{\D} \hat{\S }
\hat{\D}^\top$. The resulting global $p$-value for $H_0$ is $1 - \Prob(\max(|%
\T|) \le \max|\mathtt{|)}$ when $\T = $ has been observed. Efficient methods
for approximating the above multivariate normal and $t$ probabilities are
described in \cite{Genz1992,GenzBretz1999,BretzGenzHothorn2001} and \cite%
{GenzBretz2002}. %The procedures
%are applicable to small and moderate problems with up to $k < 100$ hypotheses.

In contrast to the global $F$- or $\chi^2$-test, the max-$t$ test $\max(|\T|)
$ also provide information, which of the $k$ individual null hypotheses $%
H_0^j, j = 1, \dots, k$ is significant as shown in the next subsection.

\subsection{Simultaneous Inference}

\label{simtest} %\paragraph{Simultaneous tests.}

We now consider testing the $k$ null hypotheses $H_0^1, \ldots, H_0^k$
individually and require that the familywise error rate, i.e., the
probability of falsely rejecting at least one true null hypothesis, is
bounded by the nominal significance level $\alpha \in (0, 1)$. In what
follows we use adjusted $p$-values to describe the decision rules. Adjusted $%
p$-values are defined as the smallest significance level for which one still
rejects an individual hypothesis $H_0^j$, given a particular multiple test
procedure. In the present context, the adjusted $p$-value for the $j$th
individual two-sided hypothesis $H_0^j: (\K \vartheta)_j = \m_j, j = 1,
\dots, k, $ is given by 
\begin{eqnarray*}
p_j = 1 - \Prob(\max(|\T|) \le |t_j|),
\end{eqnarray*}
where $t_1, \dots, t_k$ denote the observed test statistics. By
construction, we can reject an elementary null hypothesis $H_0^j$, $j= 1,
\ldots, k$, whenever the associated adjusted $p$-value is less than or equal
to the pre-specified significance level $\alpha$, i.e., $p_j \leq \alpha$.
The adjusted $p$-values are calculated from expression~(\ref{maxt}).

Similar results also hold for one-sided testing problems. The adjusted $p$%
-values for the two different directions of the alternative hypothesis are
given by 
\begin{eqnarray*}
& & H_0: \K \vartheta \ge \m \text{ vs. } H_1: \K \vartheta < \m \quad
\Rightarrow \quad p_j = 1 - \Prob(\max(\T) \le t_j) \quad \text{("less")}, \\
& & H_0: \K \vartheta \le \m \text{ vs. } H_1: \K \vartheta > \m \quad
\Rightarrow \quad p_j = 1 - \Prob(\min(\T) \ge t_j) \quad \text{("greater")}.
\end{eqnarray*}
As before, these probabilities can be calculated from 
\begin{eqnarray*}
& & \Prob(\max(\T) \le t) \cong\int\limits_{-\infty}^t \cdots
\int\limits_{-\infty}^t \varphi_k(x_1, \dots, x_k; \nu, \Cor) \, dx_1 \cdots
dx_k \quad \text{("less")}, \\
& & \Prob(\min(\T) \ge t) \cong \int\limits_{t}^\infty \cdots
\int\limits_{t}^\infty \varphi_k(x_1, \dots, x_k; \nu, \Cor) \, dx_1 \cdots
dx_k \quad \text{("greater")}.
\end{eqnarray*}
Again, we refer to \cite{Genz1992,GenzBretz1999,BretzGenzHothorn2001} and 
\cite{GenzBretz2002} for the numerical details.

%\paragraph{Simultaneous confidence intervals.}

In addition to a simultaneous test procedure, a simultaneous $(1 - 2\alpha)
\times 100\%$ confidence interval for $\K \vartheta$ is given by 
\begin{eqnarray*}
\K \hat{\vartheta} \pm q_\alpha \text{diag}(\hat{\S })^{-1/2}
\end{eqnarray*}
where $q_\alpha$ is the $1 - \alpha$ quantile of the distribution of $\T$
such that $\Prob(\max{|\T|} \le q_\alpha) \ge 1 - \alpha$. The corresponding
one-sided versions are defined analogously.

It should be noted that the simultaneous inference procedures described so
far belong to the class of single-step procedures, since a common critical
value is used for the individual tests. However, single-step procedures can
always be improved by stepwise extensions based on the closed test
procedure. That is, for a given family of null hypotheses $H_0^1, \dots,
H_0^k$, an individual hypothesis $H_0^j$ is rejected only if all
intersection hypotheses $H_J = \bigcap_{i \in J} H_0^i$ with $j \in J
\subseteq \{1, \dots, k\}$ are rejected \citep{Marcusetal1976}. Such
stepwise extensions can thus be applied to any of the methods discussed in
this paper, see for example \cite{Westfall1997} and \cite{WestfallTobias2007}%
. %%In fact, the \Rpackage{multcomp} package
%%introduced in Section~\ref{implementation} uses max-$t$
%%type statistics for each intersection hypothesis based on the
%%methods from this paper, thus accounting for stochastic
%%dependencies. Furthermore, the implementation of \Rpackage{multcomp}
%%exploits logical constraints, leading to computationally
%%efficient, yet powerful truncated closed test procedures, see
%%\cite{Westfall1997} and \cite{WestfallTobias2007}.

\section{Applications}

\label{applications}

The methodological framework described in Sections~\ref{model} and \ref%
{siminf} is very general and thus applicable to a wide range of statistical
models. Many estimation techniques, such as (restricted) maximum likelihood
and M estimates, provide at least asymptotically normal estimates of the
parameters together with consistent estimate of the covariance matrix. In
this section we illustrate the generality of the methodology by reviewing
some potential applications. Detailed numerical examples are discussed in
Section~\ref{illustrations}. In what follows, we assume $\m = 0$ only for
the sake of simplicity. The next paragraphs highlight a subjective selection
of some special cases of practical importance.

\paragraph{Multiple Linear Regression.}

In standard regression models the observations $\Z_i$ of subject $i=1,
\ldots, n$ consist of a response variable $Y_i$ and a vector of covariates $%
\X_i = (X_{i1}, \dots, X_{iq})$, such that $\Z_i = (Y_i, \X_i)$ and $p = q +
1$. The response is modelled by a linear combination of the covariates with
normal error $\varepsilon_i$ and constant variance $\sigma^2$, 
\begin{eqnarray*}
Y_i = \beta_0 + \sum_{i = 1}^q \beta_i X_{ij} + \sigma \varepsilon_i,
\end{eqnarray*}
where $\varepsilon = (\varepsilon_1, \dots, \varepsilon_n)^\top \sim \N_n(0, 
\mathbf{I}_n).$ The parameter vector of interest is $\vartheta = (\beta_0,
\beta_1, \dots, \beta_q)$, which is usually estimated by 
\begin{eqnarray*}
\hat{\vartheta} = \left(\X^\top\X\right)^{-1} \X^\top \Y \sim \N%
_{q+1}\left(\vartheta, \sigma^2 \left(\X^\top\X\right)^{-1}\right),
\end{eqnarray*}
where $\Y = (Y_1, \dots, Y_n)$ denotes the response vector and $\X
= (1, (X_{ij}))_{ij}$ denotes the design matrix, $i = 1, \dots, n, j = 1,
\dots, q$. Thus, for every matrix $\K \in \R^{k,q+1}$ of constants
determining the experimental questions of interest we have 
\begin{eqnarray*}
\K \hat{\vartheta} \sim \N_k(\K \vartheta, \sigma^2 \K \left(\X^\top\X%
\right)^{-1} \K^\top).
\end{eqnarray*}
Under the null hypothesis $\K \vartheta = 0$ the standardized test
statistics 
\begin{eqnarray*}
\T = \hat{\D} \K \hat{\vartheta} \sim t_{q+1}(n - q, \Cor),
\end{eqnarray*}
where $\hat{\D}$ is the diagonal matrix of the inverse estimated standard
deviations of $\K \hat{\vartheta}$ and $\Cor$ is the correlation matrix as
given in Section~\ref{siminf}. The body fat prediction example presented in
Subsection \ref{bodyfat} illustrates the application of simultaneous
inference procedures in the context of variable selection in linear
regression models.

\paragraph{One-way ANOVA.}

Consider a one-way ANOVA model for a factor measured at $q$ levels with a
continuous response 
\begin{eqnarray}  \label{one-way}
Y_{ij} = \mu + \gamma_{j} + \varepsilon_{ij}
\end{eqnarray}
and independent normal errors $\varepsilon_{ij} \sim \N_1(0, \sigma^2), j =
1, \dots, q, i = 1, \dots, n_j$. Note that the model description in (\ref%
{one-way}) is overparameterized. A standard approach is to consider a
suitable re-parametrization. The so-called "treatment contrast" vector $%
\vartheta = (\mu, \gamma_2 - \gamma_1, \gamma_3 - \gamma_1, \dots, \gamma_q
- \gamma_1)$ is, for example, the default re-parametrization used in \RR.

Many classical multiple comparison procedures can be embedded into this
framework, including Dunnett's many-to-one comparisons and Tukey's
all-pairwise differences. For Dunnett's procedures, the differences $%
\gamma_i - \gamma_1$ are tested for all $i=2, \ldots, i$, where $\gamma_1$
denotes the mean treatment effect of a control group. In the notation from
Section~\ref{model} we thus have 
\begin{eqnarray*}
\K_\text{Dunnett} = (0, \diag(q))
\end{eqnarray*}
resulting in the linear functions 
\begin{eqnarray*}
\K_\text{Dunnett} \vartheta = (\gamma_2 - \gamma_1, \gamma_3 - \gamma_1,
\dots, \gamma_q - \gamma_1)
\end{eqnarray*}
of interest. For Tukey's procedure, the interest is in all-pairwise
comparisons of the parameters $\gamma_1, \dots, \gamma_q$. For $q = 3$, for
example, we have 
\begin{eqnarray*}
\K_\text{Tukey} = \left( 
\begin{array}{rrr}
0 & 1 & 0 \\ 
0 & 0 & 1 \\ 
0 & 1 & -1%
\end{array}
\right)
\end{eqnarray*}
with 
\begin{eqnarray*}
\K_\text{Tukey} \vartheta = (\gamma_2 - \gamma_1, \gamma_3 - \gamma_1,
\gamma_2 - \gamma_3).
\end{eqnarray*}

Many further multiple comparison procedures have been investigated in the
past, which all fit into this framework. We refer to Bretz et al. (2001) for
a related comprehensive list. Note that under the standard ANOVA assumptions
of i.i.d. normal errors with constant variance the vector of test statistics 
$\T$ follows a multivariate $t$ distribution. Thus, related simultaneous
tests and confidence intervals do not rely on asymptotics and can be
computed analytically instead, as shown in Section~\ref{siminf}.

To illustrate simultaneous inference procedures in one-way ANOVA models, we
consider all pairwise comparisons of expression levels for various genetic
conditions of alcoholism in Subsection~\ref{alpha}.

\paragraph{Further parametric models.}

In \emph{generalized linear models}, the exact distribution of the parameter
estimates is usually unknown and thus the asymptotic normal distribution is
the basis for all inference procedures. When we are interested in inference
about model parameters corresponding to levels of a certain factor, the same
multiple comparison procedures as sketched above are available. \emph{Linear
and non-linear mixed effects} models fitted by restricted maximum-likelihood
provide the data analyst with asymptotically normal errors and a consistent
covariance matrix as well so that all assumptions of our framework are met
and one can set up simultaneous inference procedures for these models as
well. The same is true for the \emph{Cox model} or other parametric survival
models such as the \emph{Weibull survival model}.

We use logistic regression models to estimated the probability of suffering
from Alzheimer's disease in Subsection~\ref{alzheimer}, compare several risk
factors for survival of leukemia patients by means of a Weibull model in
Subsection~\ref{AML} and obtain probability estimates of deer browsing for
various tree species from mixed models in Subsection~\ref{forest}.

\paragraph{Robust simultaneous inference.}

Yet another application is to use robust variants of the previously
discussed statistical models. One possibility is to consider the use of
sandwich estimators $\hat{\Sigma}$ for the covariance matrix $\Sigma$ when,
for example, the variance homogeneity assumption is violated. An alternative
is to apply robust estimation techniques in linear models, for example S-,
M- or MM-estimation 
\citep[see][for
example]{RousseeuwLeroy2003, mfluc:Stefanski+Boos:2002, Yohai1987, mfluc:White:1994}%
, which again provide us with asymptotically normal estimates.

The reader is referred to Subsection~\ref{bodyfat} for some numerical
examples illustrating these ideas.

\section{Implementation}

\label{implementation}

The \Rpackage{multcomp} package \citep{pkg:multcomp} in \RR{} \citep{R2007}
provides a general implementation of the framework for simultaneous
inference in (semi-)\-parametric models described in Sections~\ref{model}
and~\ref{siminf}. The numerical examples in Section~\ref{illustrations} will
all be analyzed using the \Rpackage{multcomp} package. In this section we
briefly introduce the \Rpackage{multcomp} package and refer the reader to
the online documentation of the package for the technical details.

Estimated model coefficients $\hat{\vartheta}$ and their covariance matrix $%
\hat{\Sigma}$ are accessible in \RR{} with \Rcmd{coef()} and \Rcmd{vcov()}
methods available for most statistical models in \RR, such as objects of
class \Rclass{lm}, \Rclass{glm}, \Rclass{coxph}, \Rclass{nlme}, or %
\Rclass{survreg}. Having this information, the \Rcmd{glht()} function sets
up the \underline{g}eneral \underline{l}inear \underline{h}ypo\underline{t}%
hesis for a model `\Robject{model}' and a representation of the matrix $\K$
(via its \Robject{linfct} argument): 
\begin{Sinput}
glht(model, linfct, alternative = c("two.sided", "less", "greater"),
     rhs = 0, ...)
\end{Sinput}
The two remaining arguments \Rarg{alternative} and \Rarg{rhs} define the
direction of the alternative (see Section~\ref{siminf}) and $\m$,
respectively.

The matrix $\K$ can be described in three different ways:

\begin{itemize}
\item by a matrix with \Rcmd{length(coef(model))} columns, or

\item by an expression or character vector giving a symbolic description  of
the linear functions of interest, or

\item by an object of class \Rclass{mcp}  (for \underline{m}ultiple 
\underline{c}omparison \underline{p}rocedure).
\end{itemize}

The last alternative is convenient when contrasts of factor levels are to be
compared and the model contrasts used to define the design matrix of the
model have to be taken into account. The \Rcmd{mcp()} function takes the
name of the factor to be tested as an argument as well as a character
defining the type of comparisons as its value. For example, \Rcmd{mcp(type =
"Tukey")} sets up a matrix $\K$ for Tukey's all-pairwise differences among
the levels of the factor \Robject{type}, which has to appear on right hand
side of the model formula of \Robject{model}. In this particular case, we
need to assume that \Rcmd{model.frame()} and \Rcmd{model.matrix()} methods
for \Robject{model} are available as well.

Objects of class \Rclass{glht} returned by \Rcmd{glht()} include %
\Rcmd{coef()} and \Rcmd{vcov()} methods to compute $\K \hat{\vartheta}$ and $%
\hat{\S }$. Furthermore, a \Rcmd{summary()} method is available to perform
different tests (max $t$, $\chi^2$ and $F$-tests) and $p$-value adjustments,
including those taking logical constraints into account \citep{Shaffer1986,
Westfall1997}. In addition, the \Rcmd{confint()} method applied to objects
of class \Rclass{glht} returns simultaneous confidence intervals and allows
for a graphical representation of the results. The numerical accuracy of
adjusted $p$-values and simultaneous confidence intervals implemented in %
\Rpackage{multcomp} is continuously checked against results reported by \cite%
{Westfall1999}.

\input{illustrations}

\input{trees}

\section{Conclusion}

Multiple comparisons in linear models have been in use for a long time, see 
\cite{HochbergTamhane1987}, \cite{Hsu1996}, and \cite{Bretzetal2008}. In
this paper we have extended the theory to a broader class of parametric and
semi-parametric statistical models, which allows a unified treatise of
multiple comparisons and other simultaneous inference procedures in
generalized linear models, mixed models, survival models, robust models etc.
In essence, all what is required is a parameter estimate $\hat{\vartheta}$
following an asymptotic multivariate normal distribution, and a consistent
estimate $\hat{\Sigma}$ of its covariance matrix. Standard software packages
can be used to compute these quantities. As shown in this paper, these
quantities are sufficient to derive powerful simultaneous inference
procedures, which are tailored to the experimental questions under
investigation. Therefore, honest decisions based on simultaenous inference
procedures maintaining a pre-specified family-wise error rate can now be
based on almost all classical and modern statistical models.

The examples presented in Section~\ref{illustrations} illustrate two facts.
At first, the presented approach helps to formulate simultaneous inference
procedures in situations that were previously hard to deal with and, at
second, a flexible open-source implementation offers tools to actually
perform such procedures rather easily. With the \Rpackage{multcomp} package,
freely available from \url{http://CRAN.R-project.org}, honest simultaneous
inference is only two commands away. 
\marginpar{
vignette}

\bibliographystyle{biometrika}
\bibliography{references}

\end{document}
